\documentclass[12pt]{article}
\usepackage[utf8]{inputenc}
\usepackage[english]{babel}
\usepackage[dotinlabels]{titletoc}
\usepackage[nottoc]{tocbibind}
\usepackage{mathptmx}
\usepackage{amsmath, amssymb}
\usepackage{geometry, titlesec, setspace}
\usepackage{graphicx, caption}
\usepackage{hyperref, natbib}
% Journals
\newcommand{\aap}{A\&A}
\newcommand{\aj}{AJ}
\newcommand{\apj}{ApJ}
\newcommand{\araa}{ARA\&A}
\newcommand{\mnras}{MNRAS}
\newcommand{\pasp}{PASP}
\newcommand{\physrep}{PhR}
\newcommand{\rpph}{RPPh}

% Formatting and paths
\hypersetup{
	colorlinks=true,
	citecolor=blue,
	filecolor=blue,
	linkcolor=blue,
	urlcolor=blue
}
\graphicspath{{figs/}}
\captionsetup[figure]{labelfont={bf}, font=small, name={Figure}, labelsep=period}
\captionsetup[table]{labelfont={bf}, font=small, name={Table}, labelsep=period}
\titleformat{\section}{\normalsize\bfseries\centering}{\thesection.}{0.75em}{}
\titleformat{\subsection}{\normalsize\itshape\centering}{\thesubsection.}{0.75em}{}
\renewcommand{\contentsname}{Table of Contents}
\renewcommand{\abstractname}{{\normalsize\textbf{Abstract}}}

% For formatting text
\newcommand{\code}[1]{{\fontfamily{qcr}\selectfont#1}}
\newcommand{\red}[1]{\textcolor{red}{#1}}

% Typesetting special symbols
\newcommand{\HI}{H\,\textsc{i}}

\begin{document}
\doublespacing
\begin{center}
Environmental Systematics and the Impact on \\ 21-cm Epoch of Reionization Measurements \\
by \\
Lily Whitler \\
has been approved \\
Spring 2019 \\[0.1\textheight]

\begingroup
\renewcommand{\arraystretch}{0.7}
\begin{tabular}{p{1cm}p{3.5in}p{1cm}}
	& \centering APPROVED: & \\
	& & \\ & & \\
	& \hrulefill & \\
	& \hfill Daniel Jacobs, Director & \\
	& & \\ & & \\
	& \hrulefill & \\
	& \hfill Judd Bowman & \\
	& & \\ & & \\
	& \hrulefill & \\
	& \hfill Adam Beardsley &
\end{tabular} \\[0.075\textheight]
\begin{tabular}{p{1cm}p{3.5in}p{1cm}}
	& \centering ACCEPTED: & \\
	& & \\ & & \\
	& \hrulefill & \\
	& \hfill Dean, Barrett, The Honors College &
\end{tabular}
\endgroup
\end{center}
\thispagestyle{empty}
\newpage
\pagenumbering{arabic}

\begin{center}
	{\Large Environmental Systematics and the Impact on \\ 21-cm Epoch of Reionization Measurements} \\
	by \\
	Lily Whitler \\[0.15\textheight]
	
	A Thesis Presented in Partial Fulfillment \\
	of the Requirements for Graduation from \\
	Barrett, the Honors College \\[0.15\textheight]
	
	Committee: \\
	Daniel Jacobs, Director \\
	Judd Bowman \\
	Adam Beardsley \\[0.2\textheight]
	
	ARIZONA STATE UNIVERSITY \\
	April 2019
\end{center}
\thispagestyle{empty}

\clearpage
\pagenumbering{roman}

\begingroup
\hypersetup{
	colorlinks=true,
	citecolor=DarkBlue,
	filecolor=black,
	linkcolor=black,
	urlcolor=DarkBlue
}
\tableofcontents
\listoffigures
\listoftables
\endgroup
\newpage

\begin{abstract}
\end{abstract}

\clearpage
\pagenumbering{arabic}

\section{Introduction} \label{sec:intro}

\subsection{A Brief History of the Universe} \label{subsec:universe}

Immediately after the Big Bang, the universe was a hot plasma of fundamental particles. Now, nearly 14 billion years later, it is populated with a rich variety of \red{objects}, from massive galaxy clusters all the way down to our own solar system and its planets. The \red{evolutionary} path of the universe from the Big Bang to now, however, is not entirely clear.

In the early universe, matter was hot and ionized. Photons scattered off of free particles, rendering the universe opaque to electromagnetic radiation. This lasted until approximately 380,000 years after the Big Bang, when the universe had expanded and cooled sufficiently for electrons to become bound to atomic nuclei during recombination. When recombination was complete, photons were able to propagate freely through space, subject only to cosmological redshift. Today, we see photons from this era as the cosmic microwave background (CMB).

Immediately following the release of the CMB and lasting until several hundred million years after the Big Bang was the cosmic Dark Ages. During this era, though photons were free to propagate, no stars, galaxies, or other sources of radiation had yet formed. The only sources of information we have from the Dark Ages are CMB photons and emission from neutral hydrogen.

The transition between the Dark Ages and the subsequent Epoch of Reionization (EoR) was marked by the \red{emergence} of the first luminous sources. During the EoR, ultraviolet (UV) radiation from these objects ionized the intergalactic medium (IGM) around them. The EoR ended when the IGM was fully ionized, thus completing the final major phase change of hydrogen in the universe.

Figure \ref{fig:timeline} shows an illustration of this evolution. The \textit{Planck} satellite and its predecessors, the \textit{Cosmic Background Explorer} and \textit{Wilkinson Microwave Anisotropy Probe}, have measured the temperature and polarization anisotropies in the CMB, providing us with a wealth of information about the early universe. On the other end of cosmic history, the Hubble Deep Fields have probed some of the oldest known galaxies. However, much of the Dark Ages and EoR still lack observational constraints. \red{I should probably have references in this little subsection...}

\begin{figure}[tb]
	\centering
	\includegraphics[width=\textwidth]{timeline.jpg}
	\caption[Timeline of the universe]{Timeline of the evolution of the universe. Image courtesy of NASA, ESA, and the Planck Collaboration.}
	\label{fig:timeline}
\end{figure}

\subsection{The Epoch of Reionization} \label{subsec:eor}

The physics during the Dark Ages was much simpler than during the EoR, since only hydrogen and CMB photons were present, but this same simplicity makes the Dark Ages more difficult to observe due to the lack of luminous sources. In contrast, observational probes of the EoR have the advantage of being able to target either the ionizing UV sources or the neutral hydrogen gas.

\red{[Talk about the theory of the EoR---ionizing bubbles, etc.]}

\subsection{The 21-cm Power Spectrum} \label{subsec:ps}

\red{[What is it?]}

\red{[Challenges---foregrounds, calibration, etc.]}

\subsection{Radio Frequency Interference} \label{subsec:rfi}

Radio frequency interference (RFI) is one such environmental contaminant.

\subsection{The Hydrogen Epoch of Reionization Array} \label{subsec:hera}

\begin{figure}[tb]
	\centering
	\includegraphics[width=0.75\textwidth]{hera.png}
	\caption[HERA as of late 2017 -- early 2018]{HERA as of late 2017 -- early 2018. HERA will observe the periods prior to and during the EoR via the redshifted 21-cm line from the IGM. Image courtesy of the South African Radio Astronomy Observatory.}
	\label{fig:hera}
\end{figure}

The Hydrogen Epoch of Reionization Array (HERA; Figure \ref{fig:hera}) is one of several radio interferometers designed to study the large-scale structure during the EoR via measurements of the redshifted 21-cm line \citep{deboer2017}---others include the Low Frequency Array \citep[LOFAR;][]{vanHaarlem2013} and the Murchison Widefield Array \citep[MWA;][]{tingay2013}. HERA both builds upon preceding instruments, including the MWA and the Donald C. Backer Precision Array for Probing the Epoch of Reionization \citep[PAPER;][]{parsons2010}, and will pave the way for future experiments such as the Square Kilometer Array \citep[SKA; e.g.,][]{mellema2013}.

When complete, HERA will be composed of 350 14-meter dishes, with a dense 320-element hexagonal core and 30 outriggers. This design means that although HERA measures approximately 61,000 visibilities, it has only 6140 unique baselines \red{[diagram of what this means?]}. This redundancy can be leveraged to calibrate the array up to degeneracies in overall gain and phase, following \cite{liu2010} and \cite{zheng2014}, and to increase sensitivity for the delay spectrum approach developed by PAPER. \red{[This needs expanding, will do it later]}

\section{Methods} \label{sec:methods}

\subsection{RFI Excision Strategies} \label{subsec:rfi_excision}

\red{Initial flagging with XRFI}

\red{Flag broadcasting with \code{hera\textunderscore pspec}}

\subsection{Calculating the Power Spectrum} \label{subsec:calc_ps}

\subsection{Modelling HERA Data} \label{subsec:modelling}

\red{Making \code{hera\textunderscore sim} talk to \code{pyuvdata}}

\section{Results}

\section{Conclusions}

\bibliographystyle{apj}
\bibliography{refs}
\end{document}