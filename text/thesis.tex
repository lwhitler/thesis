\documentclass[12pt]{article}
\usepackage[utf8]{inputenc}
\usepackage[
	left=1.25in,
	right=1.25in,
	top=1in,
	bottom=1in]{geometry}
\usepackage{mathptmx}
\usepackage[dotinlabels]{titletoc}
\usepackage[nottoc,numbib]{tocbibind}
\usepackage[svgnames]{xcolor}
\usepackage{titlesec, setspace}
\usepackage{graphicx, caption}
\usepackage{hyperref, natbib}
% Journals
\newcommand{\aap}{A\&A}
\newcommand{\aj}{AJ}
\newcommand{\apj}{ApJ}
\newcommand{\araa}{ARA\&A}
\newcommand{\mnras}{MNRAS}
\newcommand{\pasp}{PASP}
\newcommand{\physrep}{PhR}
\newcommand{\rpph}{RPPh}

% Formatting and paths
\hypersetup{
	colorlinks=true,
	citecolor=blue,
	filecolor=blue,
	linkcolor=blue,
	urlcolor=blue
}
\graphicspath{{figs/}}
\captionsetup[figure]{labelfont={bf}, font=small, name={Figure}, labelsep=period}
\captionsetup[table]{labelfont={bf}, font=small, name={Table}, labelsep=period}
\titleformat{\section}{\normalsize\bfseries\centering}{\thesection.}{0.75em}{}
\titleformat{\subsection}{\normalsize\itshape\centering}{\thesubsection.}{0.75em}{}
\renewcommand{\contentsname}{Table of Contents}
\renewcommand{\abstractname}{{\normalsize\textbf{Abstract}}}

% For formatting text
\newcommand{\code}[1]{{\fontfamily{qcr}\selectfont#1}}
\newcommand{\red}[1]{\textcolor{red}{#1}}

% Typesetting special symbols
\newcommand{\HI}{H\,\textsc{i}}

\begin{document}
\doublespacing
\begin{center}
Environmental Systematics and the Impact on \\ 21-cm Epoch of Reionization Measurements \\
by \\
Lily R. Whitler \\
has been approved \\
Spring 2019 \vspace{0.1\textheight}

\begingroup
\renewcommand{\arraystretch}{0.7}
\begin{tabular}{p{1cm}p{3.5in}p{1cm}}
	& \centering APPROVED: & \\
	& & \\ & & \\
	& \hrulefill & \\
	& \hfill Prof. Daniel Jacobs, Director & \\
	& & \\ & & \\
	& \hrulefill & \\
	& \hfill Prof. Judd Bowman & \\
	& & \\ & & \\
	& \hrulefill & \\
	& \hfill Dr. Adam Beardsley &
\end{tabular} \vspace{0.075\textheight} \\
\begin{tabular}{p{1cm}p{3.5in}p{1cm}}
	& \centering ACCEPTED: & \\
	& & \\ & & \\
	& \hrulefill & \\
	& \hfill Dean, Barrett, The Honors College &
\end{tabular}
\endgroup
\end{center}
\thispagestyle{empty}
\newpage
\pagenumbering{arabic}

\begin{center}
	Environmental Systematics and the Impact on \\ 21-cm Epoch of Reionization Measurements \\
	by \\
	Lily R. Whitler
	
	\begin{spacing}{1}
		\vspace{0.15\textheight}
		A Thesis Presented in Partial Fulfillment \\
		of the Requirements for the Degree \\
		Bachelor of Science with Honors \\
		\vspace{0.22\textheight}
		Committee: \\
		Daniel Jacobs, Director \\
		Judd Bowman \\
		Adam Beardsley
	\end{spacing}

	\vspace{0.22\textheight}
	ARIZONA STATE UNIVERSITY \\
	April 2019
\end{center}
\thispagestyle{empty}
\newpage

\tableofcontents
\listoffigures
\listoftables
\newpage

\begin{abstract}
\end{abstract}

\section{Introduction} \label{sec:intro}
\subsection{The Epoch of Reionization} \label{subsec:eor}
\cite{morales2010} \\
\cite{pritchard2012}
\subsection{The 21-cm Power Spectrum} \label{subsec:ps}
\subsection{The Hydrogen Epoch of Reionization Array} \label{subsec:hera}
\begin{figure}[h]
	\centering
	\includegraphics[width=0.75\textwidth]{hera.png}
	\caption[HERA as of late 2017 -- early 2018]{HERA as of late 2017 -- early 2018. HERA will observe large-scale structure prior to and during the EoR via the redshifted 21-cm line from the intergalactic medium. Image courtesy of the South African Radio Astronomy Observatory.}
	\label{fig:hera}
\end{figure}
\cite{deboer2017}
\subsection{Radio Frequency Interference} \label{subsec:rfi}

\section{Methods} \label{sec:methods}
\subsection{RFI Excision Strategies} \label{subsec:rfi_excision}
\subsection{Calculating the Power Spectrum} \label{subsec:calc_ps}
\subsection{Modelling HERA Data} \label{subsec:modelling}

\section{Results}

\section{Conclusion}

\bibliographystyle{apj}
\bibliography{refs}
\end{document}